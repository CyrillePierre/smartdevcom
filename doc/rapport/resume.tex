
\chapter*{Résumé}
Le but de ce projet de dernière année est de réaliser une \textbf{intelligence artificielle} capable de communiquer avec nous, cela dans le but de créer un réseau d'objets intelligents sans limite physique au niveau du support de communication de l'objet, de sa position et enfin de son type. Ce réseau permettra donc de pouvoir parler avec un objet pouvant se trouver à quelques mètres de nous, mais aussi à l'autre bout de la planète. Il nous autorisera de la même manière de pouvoir allumer la lumière de notre maison, mais aussi de pouvoir observer l'évolution de la température de notre piscine. Le dernier but de ce projet est de pouvoir être indépendant de la technologie de communication.

Afin de réaliser ce projet, il faut en premier lieu savoir ce qui caractérise un \textbf{objet intelligent}. Après avoir listé les fonctionnalités que peut avoir n'importe quel type d'objet, il faut mettre en place un \textbf{protocole} afin de pouvoir communiquer avec eux. La construction de ce protocole passe en premier lieu par la \textbf{classification} de l'ensemble des objets que l'on sera à même de pouvoir créer. Ensuite, il faut porter ce protocole sur les objets intelligents. Ces objets sont constitués d'un microcontrôleur \textbf{Mbed} et d'un ou plusieurs capteur(s) et/ou actionneur(s). Le protocole doit aussi être portée sur une application \textbf{Android}. Cette application permet d'utiliser la \textbf{reconnaissance vocale} d'Android, afin de pouvoir interagir avec les objets autour de nous, de la manière la plus naturelle possible.
	
   \paragraph{Mots clés}: Intelligence artificielle, objet intelligent, protocole, classification, Mbed, Android, reconnaissance vocale.
		% ...
   
\clearemptydoublepage
   
 \chapter*{Abstract}
The aim of this last year's project, is to realize an \textbf{artificial intelligence} able of communicating with us, in order to create a network of smart objects without any physical limit about the communication support, his position and finally his type.

Looking forward to that goal, in the first place, one must know what define a \textbf{smart object}. Having listed the features that can have any type of object, one must set a \textbf{protocol} to communicate with them. In the first place, the building of this protocol is a \textbf{classification} of the whole objects we are able to create. Then, one must brings this protocol on the smart objects. These objects are made of a \textbf{Mbed} microchip, and one or several sensors and/or actuators. The protocol must be carried on an \textbf{Android} application aswell. This application allows us to use Android \textbf{voice recognition}, in order to interact with the objects around us, in the most natural way.
 
   \paragraph{Keywords:}Artificial intelligence, smart object, protocol, classification, Mbed, Android, voice recognition.
