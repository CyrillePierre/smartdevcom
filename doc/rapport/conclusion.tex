\chapter*{Conclusion}

Le but de ce projet était de mettre en place un réseau d'objets connectés équipés de capteurs et
d'actionneurs dédiés à la domotique. Mais pour pouvoir réaliser ce genre de réseau, il est 
indispensable de mettre en place un protocole de communication pour que tous les objets se
comprennent. Étant donné que souhaitions avoir aucune contrainte lors de la création des objets
connectés, il a fallu écrire un protocole qui soit le plus générique et dynamique possible. C'est
pourquoi la tâche la plus importante du projet a été la mise en place du protocole.
Actuellement, celui-ci est opérationnel bien qu'il n'intègre pas encore toute la dynamique que
nous souhaitions.

Ce projet a donc nécessité une grande partie de réflexion et d'analyse pour concevoir un protocole 
générique. Il a requis beaucoup de connaissances dans le domaine du réseau mais également un peu
dans le domaine des statistiques pour l'organisation des types d'actions. Mais le projet n'a pas été
uniquement théorique. L'implémentation du protocole et la création d'objets ont fait intervenir des
technologies assez récente comme la norme C++14 pour le protocole, la cross-compilation avec Docker 
pour programmer sur ARM, JNI pour intégrer du code natif sous Android, des outils de programmation
temps réel ainsi que des bibliothèques d'Android et d'Mbed. Ce projet a donc été très enrichissant 
d'une part, pour les nombreux domaines de l'informatique embarqué qu'il a fait intervenir et d'autre
part, pour la fusion et la concrétisation de beaucoup de connaissances que l'on a acquise à l'ISIMA.
Cet enrichissement ne va cependant pas s'arrêter là puisque l'étape suivante est maintenant la 
création de vrais objets qui utiliseront le travail qui a été réalisé dans ce projet.
