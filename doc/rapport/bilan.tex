\chapter*{Perspectives}

Beaucoup de choses ont étaient faites dans ce projet, mais il reste encore beaucoup de choses qui sont prévus 
afin de l'améliorer.

\section*{Dynamicité du protocole}
	Avec l'ensemble des outils qui a été défini par les protocoles VIP, VARP, VDHCP et SDCP. Il 
	est théoriquement possible de créer tous les objets que l'on peut imaginer. Mais le protocole
	n'est pas encore complètement implémenté. Il manque encore la gestion dynamique des tables de
	routage, la création du protocole VDHCP (pour l'affectation automatique d'adresse VIP) et la 
	gestion des interfaces Wifi. Si l'on souhaite pouvoir profiter pleinement des fonctionnalités
	du protocole, il est donc indispensable de l'implémenter complètement pour bénéficier d'un
	réseau virtuel dynamique.

\section*{Création d'un réseau virtuel}
	La plus grande partie du projet a été consacrée à la réalisation du protocole. Mais la partie
	la plus intéressante est cependant lorsque l'on commence à créer un ensemble d'objets et de les
	mettre en réseau. On atteint alors une étape beaucoup plus visuelle et donc beaucoup agréable 
	pour travailler. Malheureusement il était indispensable de mettre d'abord en place les briques
	de base du protocole avant de pouvoir commencer un objet. Mais une fois ces problèmes
	réglés, il sera alors possible d'équiper entièrement un ensemble de maisons en domotique, afin de 
	créer ce réseau virtuel.

\section*{Ajout d'une intelligence}
	Une fois qu'a été mis en place un réseau virtuel contenant un ensemble suffisant d'objets 
	connectés, on a alors la possiblité d'obtenir des informations sur la maison ou d'effectuer
	certaines actions. Mais à ce niveau là, on peut alors monter d'un cran l'intelligence de la
	maison en créant des objets capables d'analyser les données des autres objets et de pouvoir
	effectuer des actions en conséquence. Cela signifie que l'on peut mettre en place une sorte
	d'intelligence artificielle dans certains objets. La maison serait alors capable d'anticiper
	certaines des requêtes des utilisateurs.

\section*{Création d'un logiciel intégré}
	Ce logiciel permettra d'automatiser la création de tous les objets connectés possibles. Il 
	contiendra une base de données de tous les objets que l'on peut créer, avec la liste de leurs 
	différents 	capteurs et actionneurs. Cette base de données contiendra la références de ces 
	modules afin de pouvoir les 	commander sur Internet sur des sites de références. Enfin, il 
	permettra d'envoyer le bon programme sur l'objet automatiquement.

