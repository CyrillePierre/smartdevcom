% Commande pour écrire un mot dans le lexique
% @param 1 : le mot à définir
% @param 2 : sa définition
\newcommand{\lexEntry}[2] {
	\noindent
	\parbox[t]{.3\textwidth}{\textbf{#1}}
	\hspace*{.03\textwidth}
	\parbox[t]{.65\textwidth}{#2}
	\hspace*{7pt}
}

\chapter*{Lexique}
	\lexEntry{Actionneur} {
		Dans le cadre de ce rapport, un actionneur désigne un module d'un objet connecté capable
		d'agir sur son environnement.
	}
	
	\lexEntry{Address Resolution Protocol (ARP)} {
		Le protocole ARP est un des protocoles de la famille IP. Il permet de faire le lien entre une
		adresse IP et une adresse MAC.
	}

	\lexEntry{Bluetooth} {
		Bluetooth est un standard de communication permettant l'échange de données sur une très
		courte distance (de l'ordre de 10 mètres).
	}
	
	\lexEntry{Bluetooth Low Energy (BLE)} {
		Le BLE est une amélioration du Bluetooth visant à réduire considérablement la consomation
		tout en conservant les mêmes caractéristiques que le bluetooth classique.
	}
	
	\lexEntry{Capteur} {
		Dans le cadre de ce rapport, un capteur désigne un module d'un objet connecté capable de
		mesure une grandeur physique de son environnement.
	}

	\lexEntry{Ethernet} {
		Ethernet est un protocole de communication qui permet de mettre en place
		des réseaux locaux. Le transfert de donnée se fait alors par l'envoi de
		paquets}
		
	\lexEntry{Internet Protocol (IP)} {
		IP est une famille de protocole dédié à la création d'un système d'adressage unique pour
		la mise en place de réseau d'ordinateurs.
	}
		
	\lexEntry{Support de communication} {
		Dans le cadre de ce rapport, un support de communication désigne une des technologies de
		communication existante (ex : Wifi, Bluetooth, ZigBee, Ethernet, \dots).
	}
	
	\lexEntry{UDP} {
		UDP (user datagram protocol) est un protocole réseau utilisé pour
		transférer des informations d'une application à une autre. Les données
		transférées ne sont cependant pas contrôlées.
	}
	
	\lexEntry{Voisin} {
		Un objet voisin est un objet connecté appartenant au même support de communication que
		l'objet comparé.
	}
		
	\lexEntry{Wifi} {
		Le Wifi est un ensemble de protocole de communication sans fil permettant la communication
		entre périphériques à distance moyenne (de l'ordre d'une centaine de mètres). En fonction
		du protocole, le débit de la communication peut être relativement importante.
	}
		
	\lexEntry{ZigBee} {
		ZigBee est un protocole de haut niveau permettant la communication avec de petite radios. 
		Sa consomation est faible, sa portée est d'environ 10 mètres. Il est principalement utilisé
		dans la domotique.
	}