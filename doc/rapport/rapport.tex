
\documentclass[a4paper, 11pt]{book}

\usepackage[utf8]{inputenc}      % Pour gérer correctement les accents
\usepackage[T1]{fontenc}
\usepackage[francais]{babel}     % Indentation et style français
\usepackage[babel=true]{csquotes}
\usepackage[bindingoffset=1cm, hmargin=2cm, vmargin=3cm, headheight=14pt]{geometry}
\usepackage{graphicx}
\usepackage{textcomp}      % Pour les symboles comme 'degré'
\usepackage{tocloft}       % Config des espacements des titres
\usepackage{amsmath}       % Pour les équations
\usepackage{numprint}      % Pour les nombres en style français
\usepackage{units}         % Pour \unit
\usepackage{gensymb}       % Pour le degré en math
\usepackage{subfig}        % Pour les sous-figures
\usepackage{hyperref}      % Pour faire des liens dans le document
\usepackage{fancyhdr}      % Pour l'en-tête et le pied de page
\usepackage{dsfont}		   % Pour une jolie lettre pour l'esnemble des réels
\usepackage{setspace}
\usepackage{listings} 
\usepackage[vlined, french, nofillcomment]{algorithm2e}            

\title{Les chevals sont rouges}
\author{Thomas \textsc{Impery} \\ Cyrille \textsc{Pierre}}

\setlength{\parskip}{10pt}             % Espace entre les paragraphes

\makeatletter  % Pour les variables @author et @date

% Configuration de l'en-tête et du pied de page
\fancypagestyle{plain}{%
   \renewcommand{\headrulewidth}{0pt}
   \renewcommand{\footrulewidth}{.4pt}
   \fancyhf{}
   \fancyfoot[C]{\thepage}
   \fancyfoot[L]{\@author}
   \fancyfoot[R]{\@date}}
   
\renewcommand{\headrulewidth}{.4pt}
\renewcommand{\footrulewidth}{.4pt}
\fancyhf{}
\fancyhead[L]{\leftmark}
\fancyfoot[C]{\thepage}
\fancyfoot[L]{\@author}
\fancyfoot[R]{\@date}
\pagestyle{fancy}

% Config page blanche
\let\origdoublepage\cleardoublepage
\newcommand{\clearemptydoublepage}{%
  \clearpage
  {\pagestyle{empty}\origdoublepage}%
}

\renewcommand{\baselinestretch}{1.2}   % interligne à 1.5
   

\begin{document}
	
\begin{titlepage}
   \noindent
   \setstretch{1.0}
   \begin{minipage}{.45\textwidth}
      \includegraphics[width=.8\textwidth]{img/logo_isima.png}
      \\[.5cm]
      \textbf{I}nstitut \textbf{S}upérieur \\
      d'\textbf{I}nformatique, de \\
      \textbf{M}odélisation et de \\
      leurs \textbf{A}pplications
      \\[.5cm]
      \footnotesize
      Campus des Cézaux \\
      24 avenue des Landais \\
      BP 10125 \\
      63173 AUBIERE Cedex
   \end{minipage}
   \hspace{.03\textwidth}
%    \begin{minipage}{.45\textwidth}
%       \begin{flushright}
%       \end{flushright}
%    \end{minipage}
   \\[.5cm]
   %
   \begin{center}
      Rapport d'ingénieur \\
      Projet de 3\ieme{} année \\
      Filière : Informatique des systèmes embarqués
      \rule{\textwidth}{.3pt}
      \LARGE \textbf{\textsc{\@title}}
      \rule{\textwidth}{.3pt}
      \hbox{\vspace{3cm}} \\
%       \includegraphics[width=.5\textwidth]{img/os32c.png}
   \end{center}
   {\large \textbf{\@author}}
   \\[.5cm]
   Tuteur : Michel \textsc{Cheminat} \\
   Référent : Christian \textsc{Laforest} \\
   \begin{flushright}
      \footnotesize
      Soutenance : 10/03/2016 \\
      Projet de 120 heures par personne
   \end{flushright}
\end{titlepage}

   \makeatother
   
   \clearemptydoublepage
   
   \frontmatter   % Numérotation romaine minuscule
   
   \tableofcontents
   \clearemptydoublepage
   
   \listoffigures
   \clearemptydoublepage
   
   
\chapter*{Résumé}
	% ...
	
   \paragraph{Mots clés :}
		% ...
   
\clearemptydoublepage
   
 \chapter*{Abstract}
	% ...
 
   \paragraph{Keywords:}
		% ...
   \clearemptydoublepage

   \mainmatter    % Numérotation arabe

   \chapter{Automatisation de la maison}

\section{L’intelligence des objets}
	\subsection{Définition d’un objet connecté}
Aujourd'hui les objets connectés font partis de notre quotidien, et peuvent réaliser énormément 
d'actions différentes. Globalement on peut dire qu'un objet connecté, ou l'internet des objets, représente 
l'extension d'Internet à l'ensemble des objets de notre monde physique. Cela permet de donner de 
l'intelligence à des objets, afin qu'ils puissent nous communiquer différentes informations 
(\textbf{Capteurs}), ou bien exécuter différentes choses (\textbf{Actionneurs}). Ces événements peuvent être 
réalisés manuellement ou automatiquement selon le besoin.

Malgré tout, ces objets ne sont sémantiquement pas connectés entre eux pour plein de raisons différentes. 
SmartDevCom permet de réaliser un ensemble d'objet tous connectés entre eux. Un objet connecté est constitué 
d'un ou plusieurs capteur(s) ou actionneur(s).
% 		Définition générale + notre version par rapport au projet
	\subsection{Les capteurs}
	
Les capteurs sont tous les objets qui permettent d'obtenir une information sur l'environnement. Ces capteurs 
permettent par exemple de connaître la température de la pièce, de savoir si la lumière est allumée, ou si la 
porte d'entrée est ouverte.

Ces capteurs permettent une interaction simpliste avec l'environnement puisqu'ils permettent seulement 
d'obtenir une information physique. Ainsi, l'utilisation d'un capteur est très simple, puisqu'il suffit de 
lui demander d'envoyer la valeur qu'il contient.
% 		les types de capteurs
% 		leurs caractéristiques au sein d’un objet connecté
	\subsection{Les actionneurs}
	
Les actionneurs sont tous les objets qui permettent de réaliser une action. Ces actionneurs permettent par 
exemple d'ouvrir les volets, d'allumer la climatisation, ou bien d'allumer la lumière.

Ils permettent une interaction plus prononcée avec l'environnement, comparativement aux capteurs, étant donné 
qu'ils ont besoin de paramètres pour réaliser l'action voulue. Par exemple si l'on souhaite allumer une 
lumière intelligente, il faudra préciser l'intensité lumineuse, ainsi que la couleur de celle-ci.
% 		les types de capteurs
% 		leurs caractéristiques au sein d’un objet connecté

\section{La communication des objets}
	\subsection{Présentation des technologies de communication}
Actuellement, il existe de nombreux protocoles réseaux qui permettent la communication entre les différents 
objets. Ces protocoles établissent des normes, afin de pouvoir dialoguer entre différents acteurs.

\includegraphics{img/tableau-total.png} 

Tous ces protocoles ont leur avantages et leur inconvénients. Malgré tous ces protocoles, on peut voir qu'il 
existe principalement deux familles :

\paragraph{Les protocoles très énergivores}qui permettent de faire transiter des données avec un débit très 
élevé. Cette cadence est permise grâce à une fréquence trés élevé. Cette fréquence a par conséquent 
l'inconvénient de très peu passer les obstacles, et donc d'émettre sur une courte distance.

\paragraph{Les protocoles peu énergivores}qui possèdent des caractéristiques opposées. En effet ils 
permettent de faire transiter des données avec un faible débit, sur une grande distance grâce à une fréquence 
bien moindre.

C'est donc la deuxième famille qui est la plus utilisée pour l'internet des objets, puisqu'ils n'ont pas 
besoin d'envoyer beaucoup de données, ont besoin d'envoyer le plus loin possible, et surtout d'avoir le plus 
d'autonomie possible. Les protocoles les plus réprésentatifs de cette famille, sont le Bluetooth Low Energy 
(ou BLE), le Zigbee, et Z-Wave. Ces deux derniers sont relativement onéreux et compliqués à mettre en place, 
c'est pourquoi le BLE est le protocole le plus démocratisé de cette famille.

% 		wifi, bluetooth, BLE, zig-bee, …
% 		quelques caractéristiques (débit, portée, consommation)
	\subsection{Les différentes topologies de réseaux}
En plus des différents protocoles, il existe aussi des topologies différentes pour créer un réseau d'objets.

	    \subsubsection{Etoile}
\includegraphics{img/StarNetwork.svg.png}

	    \subsubsection{P2P}
\includegraphics{img/NetworkTopology-Mesh.svg.png}

	    \subsubsection{Arbre}
\includegraphics{img/TreeTopology.png}


% 		p2p, étoile, mesh, ...
% 		lien avec les technos
	\subsection{Application du réseau à la domotique}
% 		les technologies les plus adaptées à la domotique
% 		exemples d’architecture réseau dans une maison (avec ou sans mélange de technos)

\section{La centralisation de l’intelligence : Jarvis}
	\subsection{Vision macroscopique d’un système domotique}
% 		les conséquences de la connexion des objets
	\subsection{Les scénarios}
% 		explication sur l’intérêt de la connexion des objets
% 		exemples d’utilisations
	\subsection{Analyse et anticipation de requêtes}
% 		analyse plus poussée des données : statistiques, interprétations des mesures
% 		intégration d’intelligence artificielle
% 		exécution de tâches de façon autonomes
% 		les limites

   \clearemptydoublepage
   
   \chapter{Mise en place du protocole de communication}

\section{Les protocoles existants}
	\subsection{Z-Wave}
	\subsection{Alljoyn}

\section{Le réseau virtuelle}
	\subsection{Architecture du réseau}
% 		définition d’un réseau (uniquement 1 support de communication)
% 		description de la connexion des réseaux
% 		exemples d’architecture de réseau virtuel
	\subsection{Le routage}
% 		présentation du protocole VIP
% 		explication du transfert d’un message d’un support à un autre
% 		exemple concret d’un envoi de message
	\subsection{Un réseau dynamique}
% 		apparition, disparition ou déplacement d’objets dans le réseau
% 		présentation du protocole VARP
% 		exemple d’apparition d’un objet
% 		les objets mobiles

\section{La généricité des objets connectés}
% 	explication de l’intérêt de classifier les actions
	\subsection{Méthode de classification des actions sur les objets}
% 		recherche des informations caractérisant les actions
% 		algorithmes de classification
	\subsection{La structure actuelle}
% 		explication du choix de la hiérarchie actuelle d’actions
% 		les limites et les défauts de cette hiérarchie
	\subsection{Gestion dynamique d’objets inconnus}
% 		comment un objet d’un nouveau type peut s’intégrer à la hiérarchie
% 		les différentes solutions d’intégration
% 		exemple


   \clearemptydoublepage
   
   \chapter{La création des objets}

\section{Reconnaissance vocale : Application Android}
	\subsection{Etat de l’art}
	\subsection{La reconnaissance vocale}
	\subsection{Communiquer avec les protocoles BLE et WiFi}
\section{Les actionneurs}
\section{Les capteurs}


   \clearemptydoublepage
   
   \chapter*{Conclusion}

Le but de ce projet était de réaliser un réseau d'objets intelligents et dynamiques.

Cet objectif a été en parti atteint, puisque les bases du protocoles ont été écrites, permettant la 
commnunication avec ces objets. Une application Android a aussi été réalisée, afin de pouvoir communiquer 
avec ces différents objets

Ce projet nous a apporté beaucoup de choses, puisque nous avons pu travailler avec plusieurs cartes 
électroniques Mbed. Nous avons pu mettre faire un état de l'art sur les différents protocoles existant, et 
des différents supports de communication utilisés aujourd'hui pour la domotique. Cela nous a aussi permis de 
savoir comment fonctionner la reconnaissance vocale au sein d'Android, et de savoir comment interfacer 
celle-ci avec du code natif.
   \clearemptydoublepage
   
   \frontmatter   % Numérotation romaine minuscule
   % Commande pour écrire un mot dans le lexique
% @param 1 : le mot à définir
% @param 2 : sa définition
\newcommand{\lexEntry}[2] {
	\noindent
	\parbox[t]{.3\textwidth}{\textbf{#1}}
	\hspace*{.03\textwidth}
	\parbox[t]{.65\textwidth}{#2}
	\hspace*{7pt}
}

\chapter*{Lexique}
	\lexEntry{Actionneur} {
		Dans le cadre de ce rapport, un actionneur désigne un module d'un objet connecté capable
		d'agir sur son environnement.
	}
	
	\lexEntry{Address Resolution Protocol (ARP)} {
		Le protocole ARP est un des protocoles de la famille IP. Il permet de faire le lien entre une
		adresse IP et une adresse MAC.
	}

	\lexEntry{Bluetooth} {
		Bluetooth est un standard de communication permettant l'échange de données sur une très
		courte distance (de l'ordre de 10 mètres).
	}
	
	\lexEntry{Bluetooth Low Energy (BLE)} {
		Le BLE est une amélioration du Bluetooth visant à réduire considérablement la consomation
		tout en conservant les mêmes caractéristiques que le bluetooth classique.
	}
	
	\lexEntry{Capteur} {
		Dans le cadre de ce rapport, un capteur désigne un module d'un objet connecté capable de
		mesure une grandeur physique de son environnement.
	}

	\lexEntry{Ethernet} {
		Ethernet est un protocole de communication qui permet de mettre en place
		des réseaux locaux. Le transfert de donnée se fait alors par l'envoi de
		paquets}
		
	\lexEntry{Internet Protocol (IP)} {
		IP est une famille de protocole dédié à la création d'un système d'adressage unique pour
		la mise en place de réseau d'ordinateurs.
	}
		
	\lexEntry{Support de communication} {
		Dans le cadre de ce rapport, un support de communication désigne une des technologies de
		communication existante (ex : Wifi, Bluetooth, ZigBee, Ethernet, \dots).
	}
	
	\lexEntry{UDP} {
		UDP (user datagram protocol) est un protocole réseau utilisé pour
		transférer des informations d'une application à une autre. Les données
		transférées ne sont cependant pas contrôlées.
	}
	
	\lexEntry{Voisin} {
		Un objet voisin est un objet connecté appartenant au même support de communication que
		l'objet comparé.
	}
		
	\lexEntry{Wifi} {
		Le Wifi est un ensemble de protocole de communication sans fil permettant la communication
		entre périphériques à distance moyenne (de l'ordre d'une centaine de mètres). En fonction
		du protocole, le débit de la communication peut être relativement importante.
	}
		
	\lexEntry{ZigBee} {
		ZigBee est un protocole de haut niveau permettant la communication avec de petite radios. 
		Sa consomation est faible, sa portée est d'environ 10 mètres. Il est principalement utilisé
		dans la domotique.
	}
   \clearemptydoublepage
   
   \chapter*{Bibliographie}

	\noindent
	[1] Algorithme de la distance de Jaro-Winkler \\
	Disponible sur {\footnotesize	http://www.wikiwand.com/fr/Distance_de_Jaro-Winkler

	\noindent
	[2] Comment intégrer du code C++ dans du java \\
	Disponible sur 
{\footnotesize https://www3.ntu.edu.sg/home/ehchua/programming/java/JavaNativeInterface.html JNI

	\noindent
	[3] Texas Instruments, 2013, \emph{Dasheet of CC2541}. \\
	Disponible sur {\footnotesize http://www.ti.com/lit/ds/symlink/cc2541.pdf}
	
	\noindent
	[4] Blutetooth\texttrademark, 2003, 
	\emph{Bluetooth Network Encapsulation Protocol specification}. \\
	Disponible sur 
	{\footnotesize https://www.bluetooth.org/docman/handlers/DownloadDoc.ashx?doc_id=6552}
	
	\noindent 
	[5] Bolutek, 2014, \emph{CC41-A Specification}. \\
	Disponible sur 
	{\footnotesize http://img.banggood.com/file/products/20150104013145BLE-CC41-A\%20Spefication.pdf}
	
	\noindent 
	[6] Bolutek, 2014, \emph{CC41-A AT Commands}. \\
	Disponible sur 
	{\footnotesize https://halckemy.s3.amazonaws.com/uploads/document/file/94325/FE85NGOIH90OKGT.pdf}
	
	\noindent
	[9] Spécifications du support et du protocole zigbee \\
	Disponible sur {\footnotesizehttps://www.wikiwand.com/fr/ZigBee 
	
	\noindent
[10] différence entre le zigbee et zwave \\
Disponible sur 	{\footnotesize
http://www.domagency.fr/index.php/infos/faqs-domotique/item/116-quelles-sont-les-diff\%C3\%A9rences-entre-le- 
zigbee-et-le-z-wave? 
   \clearemptydoublepage
   
   \renewcommand{\thesection}{\Alph{section}}
   \setcounter{section}{0}
   
   \let\origappendix\appendix % save the existing appendix command
   \renewcommand\appendix{\clearpage\pagenumbering{Roman}\origappendix}
   
   \appendix
   
%    \chapter{Diagramme de Gantt}
%       {\centering
%       \includegraphics[width=.8\textwidth]{}}

\end{document}

