\chapter{Automatisation de la maison}

\section{L’intelligence des objets}
	\subsection{Définition d’un objet connecté}
Aujourd'hui les objets connectés font partis de notre quotidien, et peuvent réaliser énormément d'actions différentes. Globalement on peut dire qu'un objet connecté, ou l'internet des objets, représente l'extension d'Internet à l'ensemble des objets de notre monde physique. Cela permet de donner de l'intelligence à des objets, afin qu'ils puissent nous communiquer différentes informations 
(\textbf{Capteurs}), ou bien exécuter différentes choses (\textbf{Actionneurs}). Ces événements peuvent être 
réalisés manuellement ou automatiquement selon le besoin.

Malgré tout, ces objets ne sont sémantiquement pas connectés entre eux pour plein de raisons différentes. 
SmartDevCom a pour but de créer un réseau unique où l'ensemble de ces objets peut communiquer avec les 
autres. Les objets connectés déjà créés utilisent, pour la plupart, un protocole propriétaire, ce qui induit 
l'impossibilité d'une quelconque interopérabilité avec les objets d'un autre fournisseur. C'est pourquoi les 
objets connectés que nous utiliserons pour la mise en place de SmartDevCom seront l'œuvre de notre création. 
Cet objet connecté est constitué de peu d'éléments :
\begin{itemize}
 \item Un ou plusieurs capteur(s)
 \item Un ou plusieurs actionneur(s)
 \item Une source d'énergie
 \item Un microcontrôleur (Mbed)
 \item D'une ou plusieurs interface(s) réseau(x)
\end{itemize}

% 		Définition générale + notre version par rapport au projet
	\subsection{Les capteurs}
	
Les capteurs sont tous les objets qui permettent d'obtenir une information sur l'environnement. Ces capteurs 
permettent par exemple de connaître la température de la pièce, de savoir si la lumière est allumée, ou 
bien si la porte d'entrée est ouverte.

Ces capteurs permettent une interaction simpliste avec l'environnement puisqu'ils permettent seulement 
d'obtenir une information physique. Ainsi, l'utilisation d'un capteur est très simple, puisqu'il suffit de 
lui demander d'envoyer la valeur qu'il contient, pour que ce celui-ci nous l'envoie.
% 		les types de capteurs
% 		leurs caractéristiques au sein d’un objet connecté
	\subsection{Les actionneurs}
	
Les actionneurs sont tous les objets qui permettent de réaliser une action. Ces actionneurs permettent par 
exemple d'ouvrir les volets, d'allumer la climatisation, ou bien d'allumer la lumière.

Ils permettent une interaction plus prononcée avec l'environnement, comparativement aux capteurs, étant donné qu'ils ont besoin de paramètres pour réaliser l'action voulue. Par exemple si l'on souhaite allumer une lumière intelligente, il faudra préciser l'intensité lumineuse, ainsi que la couleur de celle-ci.
% 		les types de capteurs
% 		leurs caractéristiques au sein d’un objet connecté

\section{La communication des objets}
	\subsection{Présentation des technologies de communication}
Actuellement, il existe de nombreux protocoles réseaux qui permettent la communication entre les différents objets. Ces protocoles établissent des normes, afin de pouvoir dialoguer et se comprendre entre différents acteurs.

\begin{figure}[!ht]
         \centering
         \includegraphics[width=1.05\textwidth]{img/tableau-total.png}
         \caption{Les supports de communications les plus connus}
         \label{supports}
\end{figure}

Tous ces protocoles ont leurs avantages et leurs inconvénients. Malgré tous ces protocoles, on peut voir qu'il existe principalement deux familles :

\paragraph{Les protocoles très énergivores}qui permettent de faire transiter des données avec un débit très élevé. Cette cadence est permise grâce à une fréquence très élevée. L'inconvénient d'une onde dont la fréquence est très élevée est que celle-ci aura plus de mal à traverser des obstacles, et donc ne sera émise que sur une courte distance.

\paragraph{Les protocoles peu énergivores}qui possèdent des caractéristiques opposées. En effet ils permettent de faire transiter des données avec un faible débit, sur une grande distance grâce à une fréquence bien moindre.

C'est donc la deuxième famille qui est la plus utilisée pour l'internet des objets. Ceci s'explique par le fait qu'ils n'ont pas besoin d'envoyer beaucoup de données, ont besoin d'envoyer le plus loin possible, et surtout d'avoir le plus d'autonomie possible. Les protocoles les plus représentatifs de cette famille, sont le Bluetooth Low Energy (ou BLE), le Zigbee, et Z-Wave. Ces deux derniers sont relativement onéreux et compliqués à mettre en place, c'est pourquoi le BLE est le protocole le plus démocratisé de cette famille.

% 		wifi, bluetooth, BLE, zig-bee, …
% 		quelques caractéristiques (débit, portée, consommation)
	\subsection{Les différentes topologies de réseaux}
En plus des différents supports de communication, il existe aussi des topologies différentes pour créer un réseau d'objets. La topologie d'un système représente la manière dont celui-ci est organisé est agencé. Ainsi en fonction des besoins, tout comme pour les protocoles, une topologie sera plus adéquate qu'une autre. On parlera de nœud pour parler des différentes machines présentes dans le réseau, et de chemin ou route pour relier les nœuds entre eux.

	    \subsubsection{Etoile}
\begin{figure}[!ht]
         \centering
         \includegraphics[width=0.4\textwidth]{img/StarNetwork.png}
         \caption{Réseau étoile}
         \label{StarNetwork}
\end{figure}
C'est la topologie la plus commune dans les communications. Elle est aussi appelée l'architecture client/serveur dans le sens où il y une machine qui dirige tout le système, et par lequel toutes les données passent. Il s'agit de l'architecture la plus simple, puisqu'avec cette topologie, chaque client se connecte simplement au serveur. Son inconvénient majeur se situe au niveau du nœud principal. En effet si celui-ci ne fonctionne plus, le réseau entier tombe.

	    \subsubsection{P2P}
\begin{figure}[!ht]
         \centering
         \includegraphics[width=0.4\textwidth]{img/NetworkTopology-Mesh.png}
         \caption{Réseau P2P}
         \label{MeshNetwork}
\end{figure}
Cette topologie, aussi appelé réseau maillé, permet d'avoir une architecture réseau plus robuste, qu'une topologie étoile, puisqu'il existe plusieurs chemins pour arriver à un nœud. Ainsi, si un des nœuds ne fonctionnent plus, il ne perturbe que très peu la communication des autres. Elle donne aussi une plus grande sécurité au niveau de l'acheminement des paquets, puisque toutes les données ne passent pas au même endroit, il est donc plus difficile de les sniffer. Pour finir cette architecture permet d'avoir un réseau beaucoup plus grand et dynamique, puisqu'il n'existe pas de nœud central. L'inconvénient de cette architecture est qu'elle est plus dure à mettre en place, du fait qu'il existe plusieurs routes pour acheminer les données. De ce fait, il faut gérer la duplication possible des paquets, et le routage dynamique si un des nœuds n'est plus fonctionnel.

	    \subsubsection{Arbre}
\begin{figure}[!ht]
         \centering
         \includegraphics[width=0.6\textwidth]{img/TreeTopology.png}
         \caption{Réseau en arbre}
         \label{TreeNetwork}
\end{figure}
Cette architecture est simplement l'extension du réseau étoile, c'est l'architecture utilisée par Internet. Il s'agit simplement de la connexion de plusieurs réseau étoile entre eux.
% 		p2p, étoile, mesh, ...
% 		lien avec les technos
	\subsection{Application du réseau à la domotique}
Afin de pouvoir être intelligent, les objets doivent communiquer, et donc utiliser un protocole précis. 
L'ensemble de ces objets doit suivre une topologie, afin de pouvoir acheminer les données correctement. La topologie la plus utilisée pour communiquer avec eux est le P2P. Cela permet de pouvoir communiquer directement avec eux en étant à côté, et cela permet aussi de pouvoir créer un réseau d'objets à taille variable et dynamique.

% 		les technologies les plus adaptées à la domotique
% 		exemples d’architecture réseau dans une maison (avec ou sans mélange de technos)

\section{La centralisation de l’intelligence : Jarvis}
	\subsection{Vision macroscopique d’un système domotique}
Actuellement, la croissance de l'intégration des objets connectés dans notre quotidien est réellement forte. 
Dans un futur proche, nous serons entourés par un très grand nombre de ces objets. Mais il n'existe peu de systèmes aujourd'hui qui permettent une interaction intelligente avec l'ensemble de ces objets. Il existe peu de moyens simples et directs de réussir à communiquer avec l'ensemble de ces objets connectés sans devoir se connecter manuellement à eux. De plus tous ces acteurs utilisent un langage différent pour communiquer, il est donc difficile de réussir à centraliser un moyen de communication universel.

SmartDevCom permet de réaliser une intelligence semblable à Jarvis, qui est une intelligence artificielle forte, que l'on peut voir dans les films Iron Man. Cette intelligence peut communiquer avec tout, et peut même deviner ce qu'elle doit faire à partir d'informations qu'elle aura apprise elle-même au cours du temps. Avec le temps, Jarvis est capable d'apprendre est de réaliser des tâches complexes : Les scenarios.
% 		les conséquences de la connexion des objets
	\subsection{Les scénarios}
Les scénarios sont un ensemble d'actions que l'on souhaite faire en simultanée et/ou de manière séquentielle. 
Par exemple, lorsque l'on souhaite regarder un film, voici un scénario avancé que l'on pourrait réaliser :
\begin{itemize}
 \item Choix du film
 \item Choix du volume
 \item Fermeture des stores
 \item Extinction de la lumière principale
 \item Allumage petite lumière tamisée, accordée au thème du film
 \item Passage du téléphone en mode silencieux 
\end{itemize}

Grâce au système de scénario, il serait possible de réaliser l'intégralité des tâches, seulement en demandant de lancer le film voulu. Nous allons voir comment créer ce genre de scénario.
% 		explication sur l’intérêt de la connexion des objets
% 		exemples d’utilisations
	\subsection{Analyse et anticipation de requêtes}
Afin de créer l'intelligence des objets, l'utilisateur possède deux choix. Il peut créer un scénario manuellement en indiquant les tâches à réaliser parallèlement et séquentiellement. Ou alors il peut laisser l'apprentissage se faire. En effet, Jarvis peut simplement faire une étude statistique des utilisations des objets, et faire une corrélation entre eux. Ainsi il peut voir après plusieurs utilisations, que lorsque l'utilisateur réalise une tâche, il en réalise forcément d'autres. Ceci permet donc de réaliser une intelligence centrale, capable d'être personnalisée automatiquement pour son utilisateur, et de pouvoir réaliser un ensemble de tâches de manière autonome.

Il existe donc un autre type d'objet où est placée cette intelligence, qu'on l'on nommera des centrales. Ces centrales permettent à la fois de pouvoir réaliser ces scénarios, mais c'est aussi grâces à une base de données contenue dans chacune d'elle que l'on pourra communiquer avec le bon objet.
% 		analyse plus poussée des données : statistiques, interprétations des mesures
% 		intégration d’intelligence artificielle
% 		exécution de tâches de façon autonomes
% 		les limites
