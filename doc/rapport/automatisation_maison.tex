\chapter{Automatisation de la maison}

\section{L’intelligence des objets}
	\subsection{Définition d’un objet connecté}
% 		Définition générale + notre version par rapport au projet
	\subsection{Les capteurs}
% 		les types de capteurs
% 		leurs caractéristiques au sein d’un objet connecté
	\subsection{Les actionneurs}
% 		les types de capteurs
% 		leurs caractéristiques au sein d’un objet connecté

\section{La communication des objets}
	\subsection{Présentation des technologies de communication}
% 		wifi, bluetooth, BLE, zig-bee, …
% 		quelques caractéristiques (débit, portée, consommation)
	\subsection{Les différentes topologies de réseaux}
% 		p2p, étoile, mesh, ...
% 		lien avec les technos
	\subsection{Application du réseau à la domotique}
% 		les technologies les plus adaptées à la domotique
% 		exemples d’architecture réseau dans une maison (avec ou sans mélange de technos)

\section{La centralisation de l’intelligence : Jarvis}
	\subsection{Vision macroscopique d’un système domotique}
% 		les conséquences de la connexion des objets
	\subsection{Les scénarios}
% 		explication sur l’intérêt de la connexion des objets
% 		exemples d’utilisations
	\subsection{Analyse et anticipation de requêtes}
% 		analyse plus poussée des données : statistiques, interprétations des mesures
% 		intégration d’intelligence artificielle
% 		exécution de tâches de façon autonomes
% 		les limites
