\chapter*{Introduction}

Aujourd'hui les objets connectés sont partout. Dans notre frigo, dans notre portable, et même dans notre 
brosse à dents. Tous ces systèmes sont capables de parler, mais tous avec un langage différent, impossible à 
comprendre pour les autres. Tous ces objets ne sont que des entités dispersées, sans intelligence propre, là 
où normalement nous parlons d'objets intelligents. Tous les constructeurs créent un protocole différent pour 
faire communiquer leurs propres objets, impliquant une non compabilité avec les objets d'un concurrent, ce qui 
est préjudiciable pour les gens souhaitant acquérir un panel d'objets de diverses marques. Ainsi il serait 
dans l'intérêt des utilisateurs d'avoir un moyen de redonner l'intelligence à ses objets, et de réussir à tous 
les faire communiquer entre eux, peu importe son origine.

De nos jours, Internet est le réseau le plus connu, et un des réseau des plus abouti. Nous parlons d'Internet 
des objets pour parler des objets connectés au sens large, mais ceux-ci n'utilisent pas encore de protocole 
qui permettent un acheminement des données tel que Internet peut le faire. Ainsi notre projet se basera sur 
les travaux qu'Internet a pu apporter, afin de réaliser un protocole similaire, dédié à l'Internet des 
objets. Cela dans le but de créer un réseau virtuel se basant sur un réseau physique déjà existant, grâce à 
Internet lui-même, et des technologies tel que le Bluetooth, afin de relier tous nos objets entre eux. 

Ainsi vient donc la question de comment réaliser un réseau pareil. Afin de répondre à cela nous verrons 
ce qui caractérise véritablement un objet intelligent, puis nous verrons comment mettre en place un 
protocole générique capable de réaliser la communication de ces objets. Pour finir nous verrons la création 
de ces différents objets, ainsi que de leurs caractéristiques.