\chapter{Mise en place du protocole de communication}

\section{Les protocoles existants}
Il existe déjà des protocoles dédiés à la domotiques. Comme nous l'avons vu avant ces protocoles ont les 
caractérisques suivantes :
\begin{itemize}
 \item Consommation d'énergie très faible
 \item Portée assez grande
 \item Topologie meshée
\end{itemize}

	\subsection{Zigbee}
	\subsection{Z-Wave}
	\subsection{Alljoyn}

\section{Le réseau virtuelle}
	\subsection{Architecture du réseau}
% 		définition d’un réseau (uniquement 1 support de communication)
% 		description de la connexion des réseaux
% 		exemples d’architecture de réseau virtuel
	\subsection{Le routage}
% 		présentation du protocole VIP
% 		explication du transfert d’un message d’un support à un autre
% 		exemple concret d’un envoi de message
	\subsection{Un réseau dynamique}
% 		apparition, disparition ou déplacement d’objets dans le réseau
% 		présentation du protocole VARP
% 		exemple d’apparition d’un objet
% 		les objets mobiles

\section{La généricité des objets connectés}
% 	explication de l’intérêt de classifier les actions
	\subsection{Méthode de classification des actions sur les objets}
% 		recherche des informations caractérisant les actions
% 		algorithmes de classification
	\subsection{La structure actuelle}
% 		explication du choix de la hiérarchie actuelle d’actions
% 		les limites et les défauts de cette hiérarchie
	\subsection{Gestion dynamique d’objets inconnus}
% 		comment un objet d’un nouveau type peut s’intégrer à la hiérarchie
% 		les différentes solutions d’intégration
% 		exemple

